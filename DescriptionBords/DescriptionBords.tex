\documentclass[a4paper,11pt]{article} 
\usepackage[french]{babel}
\usepackage[utf8]{inputenc}

\usepackage{graphicx}
\usepackage{fancyhdr}
\usepackage{lastpage}
\usepackage{amsmath}
\usepackage{xspace}
\usepackage{textcomp}

\usepackage[final]{pdfpages} 

\usepackage{hyperref}

\usepackage[top=20mm, bottom=20mm, left=25mm, right=25mm]{geometry}

\pagestyle{fancy}

\usepackage{helvet}
\usepackage{bbm}

\usepackage{verbatim}
\usepackage{amsmath}
\usepackage[table]{xcolor}
\definecolor{bleugris}{rgb}{.2,.4,.5}

\definecolor{colKeys}{rgb}{0,0,1} 
\definecolor{colIdentifier}{rgb}{0,0,0} 
\definecolor{colComments}{rgb}{0,0.5,1} 
\definecolor{colString}{rgb}{0.6,0.1,0.1} 

\usepackage{listings}

% Permet l'ajout de code par insertion du fichier le contenant
% Les arguments sont :
% $1 : nom du fichier à inclure
% $2 : le type de langage (C++, C, Java ...)
\newcommand{\addCode}[2]{%

  % Configuration de la coloration syntaxique du code
  \definecolor{colKeys}{rgb}{0,0,1}
  \definecolor{colIdentifier}{rgb}{0,0,0}
  \definecolor{colComments}{rgb}{0,0.5,1}
  \definecolor{colString}{rgb}{0.6,0.1,0.1}

  % Configuration des options 
  \lstset{%
    language = #2,%
    identifierstyle=\color{colIdentifier},%
    basicstyle=\ttfamily\scriptsize, %
    keywordstyle=\color{colKeys},%
    stringstyle=\color{colString},%
    commentstyle=\color{colComments},%
    columns = flexible,%
    %tabsize = 8,%
    showspaces = false,%
    numbers = left, numberstyle=\tiny,%
    frame = single,frameround=tttt,%
    breaklines = true, breakautoindent = true,%
    captionpos = b,%
    xrightmargin=10mm, xleftmargin = 15mm, framexleftmargin = 7mm,%
  }%
    \begin{center}
    \lstinputlisting{#1}
    \end{center}
}

\newcommand{\nTitle}[1]{%
	\clearpage
	\vspace*{\fill}		%
	\begin{center}	%
		\part{#1}		%
	\end{center}
	\vspace*{\fill}		%
	\clearpage
}

\newenvironment{nAbstract} 		%
{ 								%
	\newpage 					% 
	\vspace*{\fill}				%
	\begin{center}			 	%
		\begin{abstract}		%
}{								%
		\end{abstract}			%
	\end{center}				%
	\vspace*{\fill}				%
	\newpage					%
}


\newcommand{\nClass}[1]{{\color{bleugris}{\textsl{\textbf{#1}}}}}
\newcommand{\nParameter}[1]{{\color{gray}{\textbf{#1}}}}
\newcommand{\nMethod}[1]{{\color{gray}{\textbf{#1}}}}
\newcommand{\nConstant}[1]{\texttt{\uppercase{#1}}}
\newcommand{\nKeyword}[1]{\textsl{\textbf{#1}}}

% Conversion nombres arabes / romain
\makeatletter
\newcommand{\rmnum}[1]{\romannumeral #1}
\newcommand{\Rmnum}[1]{\expandafter\@slowromancap\romannumeral #1@}
\makeatother

\setlength{\headheight}{14pt}


\title{Dictionnaire de données}
\author{H4212}
\lhead{Dictionnaire de données}
\rhead{}
\cfoot{\thepage\ de \pageref{LastPage}}

\begin{document}
\maketitle
\newpage

\section{Actionneurs}

    \paragraph{Motorisation du sol :}
        Système de remonté et de descente de la surface de la piscine.

    \paragraph{Perméabilité du sol :}
        Ensemble d'ouverture mécaniques au sol permettant de laisser passer l'eau lors de la remonté et la descente.

    \paragraph{Lampes d'éclairage :}
        Ensemble de lampes située sur le contour de la piscine afin de l'éclairer.

    \paragraph{Injecteur de produits nettoyants :}
	    Pompe permettant de verser du produit nettoyant dans la piscine.

    \paragraph{Pompe de renouvellement de l'eau :}
	    Pompe permettant de gérer l'arriver d'eau dans la pisicine.

    \paragraph{Valve d'evacuation de l'eau :}
	    Valve permettant de gérer l'évacuation de l'eau.

    \paragraph{Resistances de chauffage :}
        Système de chauffage permettant de chauffer l'eau dans le sol de la piscine (sous le sol amovible).

\section{Capteurs}

    \paragraph{Capteurs de détéction des objets au sol :}
        Ensemble de capteurs monté en parallèles qui indique de manière binaire si un objet est posé sur la surface du sol amovible.

    \paragraph{Capteur de position du sol :}
        Capteur de position servant à déterminer la position du sol amovible.

    \paragraph{Capteur de température de la piscine :}
        Capteurs permettant d'acquérire la température de l'eau dans la piscine.
        Il est situé à mi-hauteur.

    \paragraph{Capteur d'humidité :}
        Capteur permettant d'acquérire l'humidité au niveau du sol du salon.

    \paragraph{Capteur de qualité de l'eau :}
        Capteur permettant d'acquérire la qualité de l'eau dans la piscine.

    \paragraph{Capteur de niveau de produit :}
        Capteur permettant d'acquérire la quantité de produit néttoyant dans la cuve de produit nettoyant.

    \paragraph{Capteur de niveau de l'eau :}
	    Capteur permettant d'acquérire le niveau de l'eau dans la piscine.


\section{Indicateurs}


    \paragraph{Indication de la température de l'eau :}
        Affichage permettant d'indiquer à l'utilisateur la température de l'eau. 

    \paragraph{Indication de l'humidite :}
        Affichage permettant d'indiquer à l'utilisateur l'humidité de la pièce.

    \paragraph{Indication Qualité :}
	    Affichage permettant d'indiquer à l'utilisateur la qualité de l'eau.

    \paragraph{Indication du niveau de l'eau :}
	    Affichage permettant d'indiquer à l'utilisateur le niveau de l'eau.

    \paragraph{Indication du niveau de produit nettoyant :}
	    Affichage permettant d'indiquer à l'utilisateur le niveau de produit dans la cuve à produit nettoyant

    \paragraph{Alerte de la qualité de l'eau :}
	    Affichage alertant l'utilisateur lorsque la qualité de l'eau est critique et q'un nettoyage manuel est necessaire.

    \paragraph{Alerte du niveau de produit nettoyant :}
	    Affichage alertant l'utilisateur lorsque la quantité de produit nettoyante est trop basse.

    \paragraph{Alerte d'un objet sur le sol amovible :}
        Affichage alertant l'utilisateur lorsque un objet posé sur le sol amovible alors qu'une activation est demandé.
        

\section{Boutons et Potentiomètres}

    \paragraph{Activation de la piscine :}
	    Bouton permettant d'activer la piscine.

    \paragraph{Activation du Nettoyage de l'eau :}
	    Bouton permettant d'activer manuellement le nettoyage automatique de l'eau (injection de produit nettoyant).

    \paragraph{Activation du nettoyage manuel :}
        Bouton permettant d'activer le nettoyage manuel (ce bouton est situé sur le panneau d'utilisation courante).

    \paragraph{Bouton de déplacement du sol pour nettoyage :}
        Boutons permettant de monter et de descendre le sol lors du nettoyage manuel afin d'accéder à la partie inférieur de la piscine.

    \paragraph{Boutons de déplacement manuel du sol :}
        Boutons permettants de monter ou descendre manuellement le sol de la piscine pendant l'utilisation.

    \paragraph{Potentiometre de réglage de la descente du sol :}
	    Potentiomètre permettant de régler la profondeur de la piscine.

    \paragraph{Potentiometre de réglage de l'intensité des jets :}
        Potentiométre permettant de régler l'intensité des jets de remou dans la pisicine.

    \paragraph{Potentiometre d'activation de l'éclairage :}
        Potentiomètre permettant de régler la luminosité de l'éclairage de la piscine.	

    \paragraph{Potentiometre de réglage du chauffage :}
        Potentiomètre permettant de régler l'intensité du chauffage de la piscine.

    \paragraph{Règlage de l'automatisation du Nettoyage de l'eau :}
	    Potentiomètre permettant de régler le seuil de propreté en dessous duquel l'injection de produit sera automatisé.

\end{document}




